\chapter{Projekttreiber}

\section{Projektziel}

Im Rahmen des Software-Entwicklungs-Projekts {\the\year} soll ein einfach zu bedienendes Client-Server-System zum Spielen von \textit{Kings in the Corner} über ein Netzwerk implementiert werden. Die Benutzeroberfläche soll intuitiv bedienbar sein.

\section{Stakeholders}

\newcounter{sh}\setcounter{sh}{10}

\begin{description}[leftmargin=5em, style=sameline]
	
	\begin{lhp}{sh}{SH}{sh:Spieler}
		\item [Name:] Spieler
		\item [Beschreibung:] Menschliche Spieler.
		\item [Ziele/Aufgaben:] Das Spiel zu spielen.
	\end{lhp}
	
	\begin{lhp}{sh}{SH}{bsh:Spieler}
		\item [Name:] Eltern
		\item [Beschreibung:] Eltern minderjähriger Spieler.
		\item [Ziele/Aufgaben:] Um die Spieler zu kümmern, indem Eltern Spielzeit begrenzen wollen und zugriff auf sensible Inhalte begrenzen.
	\end{lhp}
	
	\begin{lhp}{sh}{SH}{bsh:gesetzgeber}
		\item [Name:] Gesetzgeber
		\item [Beschreibung:] Das Amt für Jugend und Familie.
		\item [Ziele/Aufgaben:] Die Rechte der Spieler zu schützen und zu gewähren, indem er Gesetze erstellt.
	\end{lhp}
	
	\begin{lhp}{sh}{SH}{bsh:betreuer}
		\item [Name:] Betreuer
		\item [Beschreibung:] HiWis, die SEP Projektgruppen betreuen.
		\item [Ziele/Aufgaben:] Das Entwicklungsprozess zu betreuen, zu überwachen und teilweise zu steuern als auch die Arbeit der Projektgruppen abzunehmen sowie den Studenten im Prozess Hilfe zur Verfügung zu stellen. 
	\end{lhp}
	
	\begin{lhp}{sh}{SH}{bsh:prof}
		\item [Name:] apl. Prof. Dr. Achim Ebert
		\item [Beschreibung:]  Professor, der Vorlesung veranstaltet.
		\item [Ziele/Aufgaben:]  Organistation und Leitung der Vorlesung.
	\end{lhp}
		
\end{description}

\section{Aktuelle Lage}

Aktuell wird das Spiel so gespielt, dass man sich vor Ort trifft, aber dabei ist problematisch, dass man ein Kartendeck benötigt (das nicht immer verfügbar ist) oder, dass es durch eine Pandemie nicht möglich ist sich in Gruppen zu treffen. Das Projekt wird den Spielern ermöglichen sich online Corona konform zu treffen, zu unterhalten und gemeinsam zu spielen. Die Eltern profitieren davon, dass die Kinder nicht krank werden und mit ihren Freunden spielen können.\footnote{Hier wird beschrieben, wie die fachlichen Prozesse aktuell (also vor der Implementierung) abgewickelt werden und wieso es wichtig ist, das Projekt umzusetzen. Man kann hier auch auf die Bedürfnisse einzelner Stakeholder eingehen, muss aber nicht zwingend sein.}